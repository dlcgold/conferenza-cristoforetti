\documentclass[a4paper,12pt, oneside]{book}

%\usepackage{fullpage}
\usepackage[italian]{babel}
\usepackage[utf8]{inputenc}
\usepackage{amssymb}
\usepackage{amsthm}
\usepackage{graphics}
\usepackage{amsfonts}
\usepackage{amsmath}
\usepackage{amstext}
\usepackage{engrec}
\usepackage{rotating}
\usepackage[safe,extra]{tipa}
\usepackage{showkeys}
\usepackage{multirow}
\usepackage{hyperref}
\usepackage{microtype}
\usepackage{enumerate}
\usepackage{braket}
\usepackage{marginnote}
\usepackage{pgfplots}
\usepackage{cancel}
\usepackage{polynom}
\usepackage{booktabs}
\usepackage{enumitem}
\usepackage{framed}
\usepackage{pdfpages}
\usepackage{pgfplots}
\usepackage{fancyhdr}
\pagestyle{fancy}
\fancyhead[LE,RO]{\slshape \rightmark}
\fancyhead[LO,RE]{\slshape \leftmark}
\fancyfoot[C]{\thepage}



\title{Conferenza Cristoforetti dell'11 Maggio 2018}
\author{UniShare\\\\Davide Cozzi\\\href{https://t.me/dlcgold}{@dlcgold}}
\date{}

\pgfplotsset{compat=1.13}
\begin{document}
\maketitle

\definecolor{shadecolor}{gray}{0.80}

\newtheorem{teorema}{Teorema}
\newtheorem{definizione}{Definizione}
\newtheorem{esempio}{Esempio}
\newtheorem{corollario}{Corollario}
\newtheorem{lemma}{Lemma}
\newtheorem{osservazione}{Osservazione}
\newtheorem{nota}{Nota}
\newtheorem{esercizio}{Esercizio}
\newtheorem{domanda}{Domanda}
%\tableofcontents
\renewcommand{\chaptermark}[1]{%
\markboth{\chaptername
\ \thechapter.\ #1}{}}
\renewcommand{\sectionmark}[1]{\markright{\thesection.\ #1}}
\textbf{Appunti presi in diretta senza alcuna sistemazione}
\section{primo intervento}
Nata a Milano, ma cresciuta in Trentino. Grazie al territorio, che permetteva un'ottima osservazione del cielo, a causa del basso inquinamento luminoso, e grazie ad ottimi insegnanti sceglie di frequentare il politecnico di Monaco. Finiti gli studi entra nell'esercito, nell'aereonautica, come pilota militare. L'ESA, nel 2008, fece una selezione di astronauti e decide di cogliere al volo l'occasione. Entra nel corpo astronauti dell'ESA. Dopo diversi anni di addestramento in giro per il mondo, tra Europa, USA (Huston), Russia (città delle Stelle, dove si addestrò Gagarin) e Tokyo. Nel 2014 in Kazakistan col razzo Soiouz (\textit{da cercare}) comincia il suo viaggio. Il viaggio dura 9 minuti per raggiungere l'orbita a 400km, viaggiando ad una velocità di circa 20000km/h. La navicella misurava circa 7m, non tutta adibita agli astronauti. Destinazione ISS, la stazione spaziale internazionale, che viene raggiunta dopo circa 4 orbite intorno alla terra. l'equipaggio completo è di 6 persone, 3 già a bordo e 3 che arrivano. La ISS è fatta da 2 strutture, lo stack, con moduli pressurizzate per gli astronauti e un'impalcatura esterna non pressurizzata. Tra i moduli abbiamo il modulo Columbus, col laboratorio europeo, per ricerca in assenza di peso (combustione, fluidi, biologia, scienza dei materiali etc). A bordo si hanno postazioni per lo sport, per mantenere la massa muscolare ed evitare danni all'apparato osseo. Si pratica Tapis Roulant \textit{(cercare)} e sollevamento pesi. In cambio si può godere di albe e tramonti mozzafiato, circa 16 giri al giorno, un'orbita ogni ora e mezza (senza ovviamente nuvole). Bellissime le aurore. Si possono osservare dal modulo Cupola, costruito in Italia, che gode di finestre sul pianeta, la cui vera utilità e ospitare i comandi del braccio robotico, Canadese, usato sia per assemblare i moduli del'ISS (lanciati in tempi diversi). 
 Il braccio inoltre recupera moduli cargo esterni, che non attraccano come la Soiouz; essi si chiamano Dragon e portano di tutto ( anche macchine espresso e tazzine "spaziali"). La sua permanenza è durata un mese in più per un problema tecnico. La parte di attraversamento iniziale delle atmosfere è di 5 minuti, dove la capsula diventa una palla di fuoco.Poi si attiva il paracadute e i retrorazzi di "atterraggio morbidi"...anche se l'atterraggio è tutt'altro che morbido.
 \newpage
\section{il futuro}
 La prossima missione (6 giugno 2018) si chiama Horizon. Si sta progettando un nuovo laboratorio da affiancare a Columbus e gli Infletables, moduli gonfiabili per risparmiare massa,anche se presenta problemi di sicurezza causati dai detriti spaziali.\\
 Si stanno studiando privatamente moduli per il volo orbitale, anche per turisti. In orbita bassa sta prendendo piede la Cina, con una sua stazione spaziale (da costruire tra il 2020 e il 2022). Ovviamente con collaborazioni tra Europa e Cina.\\
 L'obbiettivo resta la Luna, con un approccio più continuativo, e Marte. L'ESA ha già collaborazioni per una trivella adatta all'esplorazione del polo sud lunare, dove ci sono zone permanentemente in ombra, alla ricerca anche di risorse ( come l'acqua o del propellente) per missioni lunari più lunghe.\\
Sono già in sviluppo astronavi idonee al superamento dell'orbita bassa. Ovviamente i costi per la costruzione di un razzo idoneo sono alti, ma è in produzione da 10 anni, potrebbe effettuare un primo volo l'anno prossimo.\\
Un altro progetto è una stazione in orbita lunare, con anche missioni di atterraggio.\\
Si parla poi del Falcon IX, razzi dotati di tecnologia che ne permette il riutilizzo.\\
Si sta valutando anche l'utilizzo di aziende di terze parti per la costruzione di rover e moduli lunari. Si sta ragionando anche di sfruttare il Gateway (in orbita intorno alla luna) come punto intermezzo per il trasporto di materiali etc.\\
Passiamo ora a Marte, dove si cerca vita passata, soprattutto con analisi della roccia, per mezzo di trivelle. è in studio anche un modo per riportare in Terra dei campioni di suolo marziano.
\newpage
\section{Intervento Sandrelli}
Astrofisico (lavora presso l'istituto italiano di astrofisica, gestisce le comunicazioni), laureato a Pisa, è specializzato in fisica teorica. Prende un master in telecomunicazione scientifica a Trieste  e viene selezionato dall'ESA per divulgare la posizione dell'ESA. Ha ideato il sito \href{url}{http://avamposto42.esa.int/} col contributo della Cristoforetti per la stesura del suo diario di bordo. L'obbiettivo è raggiungere più persone possibili con la bellezza e la profondità della scienza, cosa che i social non fanno, visto che presentano i propri interessi, non qualcosa di nuovo. L'importante è il lavoro di squadra, che è servito a reggere l'obbiettivo del sito. Le notizia fanno comunque sempre filtrate, analizzate e pesate, soprattutto in un periodo così ricco di propaganda. Ha scritto un libro per bambini sul volo della Cristoforetti (\href{url}{http://www.feltrinellieditore.it/opera/opera/nello-spazio-con-samantha/}).
\section{Domande}
\textit{Persa la prima domanda}
\begin{domanda}
\textbf{Esperimenti importanti?}
Sono quelli sulla fisiologia, il corpo umano reagisce in maniera totalmente diversa in assenza di gravità
\end{domanda}
\begin{domanda}
\textbf{ci sono detriti?}
raro vederli in quanto viaggiano a velocità diverse su orbite diverse. comunque sono presenti ammaccature sulla superficie della ISS
\end{domanda}
\begin{domanda}
\textbf{A Sandrelli, ti è capitato di lavorare con squadre internazionali?}
ovviamente si, soprattutto per l'ambito dell'INAF, l'astrofisica. ora stanno collaborando per un telescopio di 40m in Cile l'ELT (extreme large telescope). Lavorano sempre in grandi consorzi, lavorando sempre in comunità.
\end{domanda}
\begin{domanda}
\textbf{utilità della politica per lo sviluppo spaziale?}
bisogna abbandonare l'idea che la scienza sia difficile ma la comprensione superficiale della scienza è semplice. 
\end{domanda}
\begin{domanda}
\textbf{navicelle con forza centrifuga per simulare la gravità}
Se ne parla poco, l'assenza di peso è gestibile quindi si stanno concentrando gli studi su altro. Inoltre avrebbe attualmente evidenti problematiche di costruzione e robustezza.
\end{domanda}

\end{document}